

\documentclass[10pt,a4paper,ragged2e]{altacv}

\geometry{left=1cm,right=9cm,marginparwidth=6.8cm,marginparsep=1.2cm,top=1.25cm,bottom=1.25cm}

\ifxetexorluatex

  \setmainfont{Carlito}
\else

  \usepackage[utf8]{inputenc}
  \usepackage[T1]{fontenc}
  \usepackage[default]{lato}
\fi

% Change the colours if you want to
\definecolor{VividPurple}{HTML}{3E0097}
\definecolor{SlateGrey}{HTML}{2E2E2E}
\definecolor{LightGrey}{HTML}{666666}
\colorlet{heading}{VividPurple}
\colorlet{accent}{VividPurple}
\colorlet{emphasis}{SlateGrey}
\colorlet{body}{LightGrey}

\renewcommand{\itemmarker}{{\small\textbullet}}
\renewcommand{\ratingmarker}{\faCircle}


\begin{document}
\name{Valério Cardoso}
\tagline{Lead Data Analytics | Data Scientist \& Machine Learning Engineer}
\photo{3cm}{myself}
\personalinfo{

  \email{valeriocardoso@outlook.com.br}
  \phone{11-9.9682-1535}
  \location{São Paulo, Brasil}
  \linkedin{in/valeriocardoso}
  \github{valeriocardoso} 
  \homepage{dev.to/valeriocardoso}
}

\begin{fullwidth}
\makecvheader
\end{fullwidth}

\AtBeginEnvironment{itemize}{\small}


\cvsection[page1sidebar]{Sobre}

Possuo 4 anos de experiência no desenvolvimento de projetos de IA envolvendo 
técnicas de Machine Learning (XGBoost, LightGBM, Random Forest, SVM) 
e Deep Learning  (LSTM, CNN, GANS, Reinforcement Learning), 
vivência em metodologias ágeis como Scrum e Kanban. 
Forte conhecimento análitico e perfil orientado a resultados com ótimo relacionamento 
interpessoal. 

\cvsection[page1sidebar]{Experiências Profissionais}

\cvevent{Tech Lead Data Analytics}{Trestto}{Fev 2019 -- Atualmente}{São Paulo, SP}
\begin{itemize}

\item Liderança e acompanhamento de equipes técnicas.

\item Desenvolvimento de projeto machine learning ponta à ponta para compreensão de
      linguagem natural em ligações telefônicas recebendo mensalmente 50 requisições.

\item Desenvolvimento de modelos de machine learning para processamento de 
      linguagem natural, análise de sentimentos e geração de diálogos.

\item Contribui na tomada de decisão para a arquitetura de microsserviços.

\item Liderei a implamentação e integração da plataforma de Machine Learning usando 
      Kuberflow.
       
\item Contribui com a iniciativa de democratização dos dados implementando ferramentas
      self service BI, como Metabase e Kibana;

%
\end{itemize}

\divider

\cvevent{Tech Lead \& Data Scientist}{Ituran RoadTrack}{Out 2017 - Fev 2019}{São Caetano do Sul, SP}
\begin{itemize}

\item  Como Tech Lead fui responsável em estruturar e coordenar o time de Dados
       no qual desempenhei o papel de Data Scientist. 

\item  Contribui na tomada de decisão sobre a arquitetura de Big Data.

\item  Desenvolvi reports analíticos usando python e bibliotecas plot.ly, seaborn e 
       matplotlib

\item  Desenvolvimento e deploy de modelos de machine learning usando Spark 
       e Spark Streaming.

\item  Desenvolvimento de modelo de machine learning para predição de Churn.

\item  Desenvolvimento de modelo de machine learning baseado em séries temporais 
       para predição de consumo.



\end{itemize}

\cvsection[page1sidebar]{Hard \& Soft Skills}

\cvtag{Python}
\cvtag{POO}
\cvtag{Machine Learning}
\cvtag{Deep Learning}
\cvtag{GCP}
\cvtag{Cloud}
\cvtag{API}
\cvtag{Devops}
\cvtag{Kubernetes}

\divider

\cvtag{Scrum}
\cvtag{Pensamento Crítico}
\cvtag{Proatividade}

\cvsection[page1sidebar]{Languages}

\cvskill{Inglês}{3}


\clearpage

\end{document}
